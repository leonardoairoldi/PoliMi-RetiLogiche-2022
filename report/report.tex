%----------------------------------------------------------------------------------------
%	PACKAGES AND DOCUMENT CONFIGURATIONS
%----------------------------------------------------------------------------------------

\documentclass[a4paper]{report}
\usepackage[utf8]{inputenc}
\usepackage[italian]{babel}
\usepackage{microtype}
\usepackage{appendix}
\usepackage{geometry}
\usepackage[T1]{fontenc}
\usepackage{siunitx} % Provides the \SI{}{} and \si{} command for typesetting SI units
\usepackage{graphicx} % Required for the inclusion of images
\usepackage{natbib} % Required to change bibliography style to APA
\usepackage{amsmath} % Required for some math elements 
\usepackage{caption}
\usepackage{tikz}
\usepackage{pdfpages}
\usepackage[hidelinks]{hyperref}

\usetikzlibrary{arrows,automata, positioning}

\usepackage{import}

%----------------------------------------------------------------------------------------
%	DOCUMENT INFORMATION
%----------------------------------------------------------------------------------------

\title{Prova Finale di Reti Logiche} % Title
\author{Leonardo Airoldi} % Author name
\date{A. A. 2021-22}

\begin{document}
\maketitle % Insert the title, author and date


\tableofcontents
\newpage



% If you wish to include an abstract, uncomment the lines below
% \begin{abstract}
% Abstract text
% \end{abstract}

%----------------------------------------------------------------------------------------
%	INTRODUZIONE
%----------------------------------------------------------------------------------------

\chapter{Introduzione}


La specifica della Prova finale (Progetto di Reti Logiche) 2021 richiede di implementare un \textbf{codificatore convoluzionale} usando il linguaggio VHDL.\\
Un codificatore convoluzionale è un dispositivo che tramite l'applicazione di un \textit{codice convoluzionale} codifica uno stream di bit in ingresso in uno stream di bit in uscita ridondante che permette a un dispositivo che legge lo stream codificato di risalire allo stream originale anche in presenza di errori applicando tecniche di \textit{error-correction}.
Codificatori convoluzionali elettronici sono molto comuni nell'ambito delle telecomunicazioni, in quanto conferiscono a una rete un certo grado di robustezza: un dispositivo ricevente può ricostruire il segnale trasmesso, che potrebbe risultare corrotto a causa del rumore sulla linea, senza richiedere un ulteriore invio dell'informazione.

\section{Specifiche}
Nel caso specifico di questo progetto, è richiesto lo sviluppo di un circuito integrato con le seguenti caratteristiche: 
\begin{enumerate}
\item il \textit{codice convoluzionale} ha \textbf{tasso di trasmissione} $\frac{1} {2}$.
\item si interfaccia con una memoria sincrona a word di 8bit e indirizzi a 16bit.
\item lo stream in ingresso è di dimensione $n$ words memorizzate a partire dall'indirizzo \texttt{0x01}.
\item la dimensione dello stream è memorizzata all'indirizzo \texttt{0x00} ed è al massimo di 255 word.
\item il dispositivo viene sempre inizializzato con un segnale di \texttt{reset}.
\item la computazione inizia al segnale di \texttt{start} e termina quando il dispositivo alza il segnale \texttt{done}. Il dispositivo deve essere in grado di gestire computazioni successive senza aver bisogno di un segnale di \texttt{reset}. Si presuppone che il segnale di \texttt{start} rimanga alto durante tutta la computazione e che non possa essere rialzato prima che il segnale di \texttt{done} sia stato riportato a \texttt{0}.
\item il dispositivo deve funzionare con un periodo di clock di almeno \texttt{100ns}
\end{enumerate}

\section{Strumenti}
Il progetto sarà sviluppato in \textbf{VHDL} (VHSIC Hardware Description Language) che permette di descrivere circuiti integrati.
Per la simulazione e la sintesi della scheda il software utilizzato è \textbf{Xilinx Vivado} (\textit{v2021.1}).
Il dispositivo sarà poi implementato su una \textbf{FPGA Artix 7 \texttt{xc7a200tfbg484-1}}.

\section{Esempio}



%----------------------------------------------------------------------------------------
%	ARCHITETTURA
%----------------------------------------------------------------------------------------

\chapter{Architettura}



%----------------------------------------------------------------------------------------
%	RISULTATI SPERIMENTALI
%----------------------------------------------------------------------------------------

\chapter{Risultati sperimentali}



%----------------------------------------------------------------------------------------
%	CONCLUSIONE
%----------------------------------------------------------------------------------------

\chapter{Conclusione}


\end{document}